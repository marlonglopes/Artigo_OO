A migração total de um sistema legado envolve, pensar em banco de dados relacionais ou orientados a objeto, ferramenta de desenvolvimento modernas , segurança (interfaces web), servidores mais robustos, treinamento da equipe de desenvolvimento (métodos novos de desenvolvimento, linguagem, acesso aos dados), treinamento da equipe de suporte ao ambiente (DBA, Servidores), reestruturação da rede de suporte a  aplicação. Treinamento dos usuários para as novas interfaces. Gestão de Mudança. 

Deve ser levado em consideração a construção do modelo conceitual da base de dados legada pensando na migração do banco de dados para
uma nova plataforma de implementação, por exemplo usando um SGBD relacional. A disponibilidade de uma documentação abstrata, na forma de um modelo conceitual dos dados do sistema existente, pode acelerar em muito o processo de migração, pois permite encurtar a etapa de modelagem de dados da novo banco de dados. \cite{heuser2001projeto}

Na busca de migrar os sistemas legados, e avaliar qual plataforma melhor se adequa melhor as necessidades de uma empresa que visa estabilidade e performace em um novo ambiente, e nesse ambiente qual banco de dados servirá melhor as necessidades das aplicações desenvolvidas em determinados domínios de aplicação. 

Para tal propósito descreveremos uma maneira de conduzir um estudo comparativo de aspéctos básicos de implementação entre os SGBD CACHE, SQL Server e Oracle.

