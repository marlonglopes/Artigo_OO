Nos últimos anos vimos muitas evoluções em plataformas já conhecidas e o aparecimento de novas plataformas propondo novas arquiteturas de soluções para sistemas computacionais. Mas ainda existem sistemas desenvolvidos com tecnologias antigas muito ultrapassadas em produção, os chamados Sistemas Legados. \cite{roberto2005}

No ano 2000 fomos alertados para o grande número de sistemas legados que ainda estão em produção e que são importantes para as empresas que os mantém, pois não se pode simplesmente desligar estes sistemas, além de ser muito difícil migrar rapidamente de plataforma.

Chamamos de Sistema Legado os sistemas desenvolvidos em linguagens e plataformas obsoletas, este conceito se aplica a um programa escrito em uma linguagem antiga como COBOL ou Fortram como também a uma plataforma de bancos de dados antiga.

Levando em consideração de que a informação hoje em dia é o bem mais valioso que se pode ter. As empresas não podem correr o risco de simplesmente continuar mantendo estes sistemas legados em produção, achando que os ambientes atuais vão de alguma maneira continuar suportando de forma estável.

