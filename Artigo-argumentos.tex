Até meados dos anos 60, os dados eram mantidos aleatoriamente em arquivos, geralmente como partes integrantes da aplicação. A partir dessa época, surgiram os primeiros Sistemas Gerenciadores de Bancos de Dados (SGBDs) comerciais, provendo armazenamento dos dados de forma independente da aplicação, contudo, sem mecanismos de acesso eficientes. \cite{boscarioli2006} 

Um percentual significativo dos sistemas de informação ainda hoje usados foram desenvolvidos ao longo dos últimos vinte anos e não utilizam bancos de dados relacionais. Os dados destes sistemas estão armazenados em arquivos de linguagens de terceira geração, como Basic, COBOL, MUMPS e outras, ou então em bancos de dados da era pré-relacional, como IMS, ADABAS, DMS-II e os SGBD do tipo CODASYL (IDMS, IDS/2,...). \cite{heuser2001projeto}

Existe um grande número de tecnologias de gerenciamento de dados de grande porte. Essas tecnologias vão de sistemas de arquivos texto até sistemas relacionais de banco de dados de grande porte. 

Independentemente do tipo de solução que uma empresa decide utilizar, o que na em certa época era a melhor solução hoje pode estar desatualizado. Nestes casos, deve ser estudado uma mudança de plataforma de armazenamento de dados. 

Existem muitos motivos para que uma empresa decida migrar de plataforma de banco de dados, e antes de migrar estas empresas devem escolher cuidadosamente a tecnologia que atenda as necessidades atuais e que atenderá as suas necessidades o maior tempo possível após a migração. 

A migração de plataforma de armazenamento é um assunto muito sério. Os dados da empresa significam conhecimento. Todos os projetos de migração embutem um nível de risco alto. Entretanto, um projeto de migração bem planejado e corretamente executado pode fazer com que a empresa possa prover soluções melhores com acesso aos dados de maneira mais rápida e otimizada.

Um sistema legado, no contexto organizacional possui vários componentes para uma discussão técnica, mas, podemos considerar o seguinte: 

- O hardware de Apoio na maioria dos sistemas obsoletos o hardware é antigo, e não existe mais fornecedor e a manutenção é cara.

- O Software de apoio, como sistema operacional, compiladores, ferramentas, etc também podem estar desatualizado. 

- O software de aplicação em sistema legado não é um único programa de aplicação, inclui geralmente vários programas. Começa com pequenos projetos de pequeno porte e ao longo do tempo varias alterações e inclusões são implementadas tornando o sistema legado muito robusto e de difícil manutenção, e muitas vezes os programadores originais nao estão mais presentes.

- Os  dados de aplicação em muitos sistemas legados possui um imenso volume de dados que se acumulou com o passar do tempo de existência do sistema. Estes dados podem conter inconsistências e muitas vezes os dados estao duplicados em diferentes bases, em função das novas implementações feitas por diferentes programadores que desconheciam o sistema totalmente.

- As regras ou processos de negócio as informações sobre os processos internos da organização, codificadas em uma linguagem e espalhadas pelos programas que fazem parte do sistema. Essas regras na maioria das vezes não estao documentadas, sendo do conhecimento tácito de gerentes, analistas e programadores antigos. Os novos sistemas dificilmente reproduzirão essas regras.

Outros fatores não menos importantes como a falta de suporte, descontinuação do banco de dados, descontinuação da ferramenta de desenvolvimento, não homologação do banco de dados para sistemas operacionais atuais. 

Todos estes motivos reforçam que a integração apenas não é suficiente, pois novos hardwares exitem atualização de sistema operacional e os novos sistemas operacionais não suportam a ferramenta de BD Legado. Logo é necessário a migração total de plataforma, Banco de Dados e Linguagem de Programação.
