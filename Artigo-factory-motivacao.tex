O padrão \emph{Factory Method} está relacionado a todos os outros padrões que conduzam
algum tipo de construção encapsulada de objetos. Além de \emph{factory} os padrões Iterator,
Singleton , Builder, Prototype e Bridge- para citar alguns – estão relacionados ao padrão
\emph{Factory Method}.

Define uma interface para a criação de um objeto, mas deixa que as subclasses
decidam qual a classe instanciar. Permite a uma classe delegar a instanciação para as
subclasses.\cite{gamma95}


\subsection{Motivação}
\label{sub:fac_motiv}

\emph{Frameworks} usam classes abstratas para definir e manter relacionamentos entre
objetos. Um \emph{framework} é muitas vezes responsável por criar esses objetos também.

Considere-se um \emph{framework} de aplicações que podem apresentar vários documentos ao
usuário. Duas abstrações chave neste \emph{framework} são as classes Application e
Documento. Ambas as classes são abstratas, e os clientes têm usalas como subclasses para perceber
suas implementações de aplicações específicas. Para criar um aplicativo de desenho, por
exemplo, podemos definir as classes e \emph{DrawingApplication} e \emph{DrawingDocument}. A
classe Application é responsável pela gestão de documentos e como criá-los assim que o usuário selecione Open ou Novo a partir de um menu, por exemplo.\cite{gamma95}

Como subclasse do documento em particular para instanciar é específico do aplicativo,
a classe Application não consegue prever que documento deve criar, a classe Application só sabe quando um novo documento deve ser criado, não o tipo do documento para criar. Isso cria um dilema: O \emph{framework} deve instanciar classes, mas só sabe sobre classes abstratas, que não pode instanciar.\cite{gamma95}

O padrão \emph{Factory Method} oferece uma solução. Ele encapsula o conhecimento de
que subclasse Documento deve criar e move este conhecimento para fora do \emph{framework}.\cite{gamma95}

\begin{center}
	\includegraphics[scale=0.40]{Figuras/image1.jpg}
	\captionof{figure}{\label{diagrama1} Representação da estrutura para criação de documentos}
	\label{fig:diagrama1}
\end{center}

As subclasses da aplicação redefinem uma operação abstrata \emph{CreateDocument} no 
aplicativo para retornar a subclasse apropriada documento. Uma vez que uma subclasse da aplicação é instanciada, ele pode então instanciar documentos sem saber sua classe. Chamamos \emph{CreateDocument} um \emph{factory method} porque é
responsável por "produzir" um objeto.\cite{gamma95}

\subsection{Aplicação}
\label{sub:fac_aplica}

Se usa \emph{Factory pattern} quando:\cite{gamma95}

\begin{itemize}
	\item uma classe (o criador) não pode antecipar a classe dos objetos que deve criar.
.	\item uma classe quer que suas subclasses possam especificar os objetos que criam.
	\item uma classe quiser localizar o conhecimento a respeito de subclasses às quais são delegadas responsabilidades específicas (em hierarquias de classes paralelas)

\end{itemize}


\subsection{Problema}
\label{sub:fac_problema}

Uma classe precisa instanciar a derivação de uma outra, mas não sabe qual. O \emph{Factory Method} permite que uma classe derivada tome essa decisão.

\subsection{Solução}
\label{sub:fac_solucao}

Uma classe derivada decide qual classe instanciar e o modo como instanciá-la.
Algumas razões existem para que não queria utilizar new diretamente. A primeira, e
mais óbvia, é não saber que classe de objeto instanciar.
Isso é comum numa aplicação bem desenhada onde variáveis são estruturadas com
base em interfaces. Assim, vários tipos de objetos diferentes podem ser associados a
essa variável. Outra razão é a necessidade de inicializar o objeto instanciado antes de ser
atribuído à variável. Importante notar
que esta inicialização não depende do ambiente onde o objeto será utilizado, mas apenas
da estrutura do próprio objeto.\cite{fact1} \cite{gamma95}



\subsection{Como utilizar}
\label{sub:fac_utilizacao}

\begin{itemize}
	\item É possível criar um objeto sem ter conhecimento algum de sua classe concreta?
	\item Esse conhecimento deve estar em alguma parte do sistema, mas não precisa estar
no cliente
	\item \emph{FactoryMethod} define uma interface comum para criar objetos
	\item O objeto específico é determinado nas diferentes implementações dessa interface
	\item O cliente do \emph{FactoryMethod} precisa saber sobre implementações concretas do
objeto criador do produto desejado

\end{itemize}


\subsection{Diagrama de classes}
\label{sub:fac_diagrama}

\begin{center}
	\includegraphics[scale=0.40]{Figuras/image2.jpg}
	\captionof{figure}{\label{diagrama2} Representação da estrutura de classes utilizando \emph{factory method}}
	\label{fig:diagrama2}
\end{center}

\begin{itemize}
	\item \emph{Product}: define a interface dos objetos criados pelo \emph{Factory Method}
	\item \emph{ConcreteProduct}: implementa a interface definida em \emph{Product}
	\item \emph{Creator}: declara o \emph{Factory Method} que retorna um objeto do tipo \emph{Product}
		\begin{itemize}
			\item  Às vezes, o \emph{Creator} não é apenas uma interface, mas pode envolver uma
classe concreta que tenha uma implementação default para o \emph{Factory Method} para retornar um objeto com algum tipo \emph{ConcreteProduct} default
			\item Pode chamar o \emph{Factory Method} para criar um produto do tipo \emph{Product}
		\end{itemize}
	\item \emph{ConcreteCreator}: faz override do \emph{Factory Method} para retornar uma instância de \emph{ConcreteProduct}
\end{itemize}

O \emph{Creator} depende de suas subclasses para definir o \emph{Factory Method} que retorne uma instância do \emph{ConcreteProduct} apropriado.

Se o cliente tem que criar de qualquer maneira uma subclasse da classe \emph{Creator}, usar subclasses pode ser uma boa solução, mas de outra maneira o cliente tem que lidar com outro ponto da evolução.

Uma desvantagem potencial do \emph{factory method} é que os clientes podem precisar ter uma subclasse \emph{Creator} apenas para criar um objeto \emph{ConcreteProduct} específico.


\subsection{Benefícios e consequencias}
\label{sub:fac_conse+benef}

\emph{Factory Method} elimina a necessidade de vincular as classes específicas do aplicativo em seu código. O código lida apenas com a interface \emph{Product}, pois pode trabalhar com qualquer classe \emph{ConcreteProduct} definida pelo usuário. Isso diminui o acoplamento, torna as classes mais flexíveis e amarra apenas a estrutura, transferindo a responsabilidade para subclasses.\cite{gamma95}

\emph{Factory Methods} eliminam a necessidade de colocar classes específicas da aplicação no código.\cite{fact2}

\begin{itemize}
	\item O código só lida com a interface \emph{Product}
	\item O código pode, portanto funcionar com qualquer classe \emph{ConcreteProduct} Provê ganchos (\emph{Hook Methods}) para subclasses.
	\item Criar objetos dentro de uma classe com um \emph{Factory Method} é sempre mais flexível do que criar objetos diretamente.
	\item O \emph{Factory Method} provê um gancho para que subclasses forneçam uma versão estendida de um objeto

	\item Criação de objetos é desacoplada do conhecimento do tipo concreto do objeto
	\item Conecta hierarquias de classe paralelas
	\item Facilita a extensibilidade

\end{itemize}

Aqui estão duas consequências adicionais do padrão \emph{Factory Method}:\cite{gamma95}

\begin{enumerate}
	\item Fornecer ganchos para subclasses. Criação de objetos dentro de uma classe com \emph{factory methods} é sempre mais flexível do que criar um objeto diretamente. \emph{Factory Method} da para subclasses um gancho para fornecer uma versão estendida
de um objeto.

No exemplo do documento, a classe \emph{Document} poderia definir um \emph{factory method} chamado \emph{CreateFileDialog} que cria uma caixa de dialogo padrão na abertura de documentos existentes.

Uma subclasse \emph{Document} pode definir uma caixa de dialogo específico do aplicativo, substituindo este \emph{factory method}. Neste caso, o \emph{factory method} não é abstrato, mas fornece implementação default.

	\item
Conecta hierarquia de classes paralelas, nos exemplos que temos considerado até agora, o \emph{factory method} é chamado apenas por \emph{Creators}. Mas isso não tem que ser o caso, os clientes podem encontrar \emph{factory methods} úteis, especialmente no caso de hierarquias de classes paralelas.

Hierarquias de classes paralelas resulta de quando uma classe delega algumas de suas responsabilidades para uma classe separada.
Considere figuras que podem ser manipuladas de forma interativa, ou seja, elas podem ser esticadas, movidas, ou giradas usando o mouse. A execução de tais interações nem sempre é fácil. Muitas vezes isso exige armazenamento e atualização das informações que registra o estado da manipulação em um determinado tempo.

Este estado é necessária apenas durante manipulação e, portanto, não precisa ser mantido no objeto. Além disso, diferentes objetos se comportam de maneira diferente quando o usuário as manipula. Para exemplo, esticar uma figura da linha poderia ter o efeito de mover um ponto final, considerando um valor de alongamento o texto pode mudar o seu espaçamento entre as linhas.

Com estas restrições, é melhor usar um objeto Manipulator separado que implementa a interação e mantém registro de qualquer
Estado de manipulação específica necessário. 
Diferentes figuras usarão diferentes manipuladores para lidar com interações específicas. O hierarquia de classes manipulatoras resultante (pelo menos parcialmente) é ilustrada na figura \ref{fig:hierarquia}:

\end{enumerate}

\begin{center}
	\includegraphics[scale=0.40]{Figuras/hier_class_par.jpg}
	\captionof{figure}{Hierarquia de classes paralelas}
	\label{fig:hierarquia}
\end{center}


A classe \emph{Figure} proporciona um \emph{factory method CreateManipulator} que permite 
os clientes a criar um \emph{Manipulator} correspondente ao da figura. As subclasses \emph{Figure}
sobreescrevem esse método para retornar uma instância da subclasse \emph{Manipulator} correto para eles. Alternativamente, a classe  \emph{Figure} pode implementar \emph{CreateManipulator} para retornar uma instancia de um \emph{Manipulator} padrão, e a classe \emph{Figure} pode simplesmente herdar esse padrão.


\subsection{Exemplos de uso}
\label{sub:fac_uso}
