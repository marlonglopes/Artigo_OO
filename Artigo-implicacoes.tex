Conforme mensionado anteriormente, um projeto de migração de plataforma de banco de dados é um assunto muito sério e deve ser realizado com bastante cuidado, levando em consideração os domínios de negocio existentes na empresa. A migração de dados é classificada em três níveis, conforme o trabalho necessário. Estes três níveis determinam a quantidade de trabalho e os elementos envolvidos na migração.

Migração de nível um inclui apenas o esquema ou definições de dados, e os dados em si. O necessário mais comum para uma migração de nível um, é quando o aplicativo esta usando uma interface que existe tanto na plataforma de banco de dados antigo quanto no novo. Além disso, o suporte a essa interface é idêntico nos dois sistemas. Não se esta usando nenhum recurso especial do banco de dados anterior. Portanto é necessário migrar somente as definição de dados e os dados em si. Também é necessário algum trabalho de migração, mas, uma vez que os dados tenham sido migrados, a aplicação estará funcionando.

Uma migração de nível dois inclui o esquema de dados e algumas alterações do código fonte, este nível pode ser visto como extensão do nível um. Imagine uma migração de nível um onde são usadas diferentes versões da mesma interface no banco de dados antigo e no novo. Neste caso migrar apenas o esquema e os dados não faz que o aplicativo funcione completamente; é necessário analisar as diferenças entre as versões. Uma vez que essas diferenças tenham sido identificadas, pode se alterar o código para que o aplicativo funcione com o banco de dados novo.

Uma migração de nível três é diferente dos dois anteriores. Nos níveis anteriores o único objetivo da migração é alterar o sistema de armazenamento de dados, isto é, fazer com que o aplicativo funcione com o novo banco de dados. Em uma migração de nível três a maioria do aplicativo é reescrito.

Para migração de uma plataforma muito ultrapassada para uma atual é necessário uma migração de nível três pois o sistema de armazenamento legado é muitas vezes em forma de arquivo ou não relacional. Os Bancos de dados chamados relacionais são os chamados atuais. Existem bancos chamados Pós-relacionais pois uma nova característica abordade hoje em dia é o mapeamento Objeto Relacional, e os chamados Bancos de Dados Orientados a Objeto implementam essa características.

Uma migração de nível três é uma medida extema e tem dois objetivos. Primeiro, como nas outras duas migrações, alterar o sistema de armazenamento de dados para uma nova plataforma. Segundo uma migração de nível três significa alterar a camada de acesso a dados de maneira que o aplicativo use recursos específicos no novo banco de dados. A migração de nível três representa maior custo e maior risco. Entretando, os benefícios do melhor desempenho e administração fazem que o trabalho e o custo faça valer a pena.

