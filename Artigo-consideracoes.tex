Este trabalho apresentou uma revisão bibliográfica sucinta sobre as questões referentes à migração ou modernização de sistemas legados, mais especificamente de bancos de dados legados. 

Foi abordada a necessidade de planejamento na atualização de bancos legados e apresentado um esquema de comparação de performace entre bancos de dados atuais.

Para que o processo estudado tenha bons resultados é necessário que a base de dados modelada tenha boa qualidade quanto a estrutura de dados definida e os tipos de dados pertencentes ao domínio testado, isso pode influênciar muito nos resultados atingidos, pois para uma base de dados real isso implica fortemente em desempenho.

Também é de grande importancia para o bom desempenho dos testes a codificação e a estruturação da arquitetura do ambiente, pois não poderá ser utilizado nenhuma funcionalidade de fabricante para não afetar a performace de maneira não reproduzível nos três ambientes.

Acreditamos que após realizados os testes, tal estudo poderá contribuir para uma futura escolha de ambiente de armazenamento de dados para uma possível migração de sistema legado.


